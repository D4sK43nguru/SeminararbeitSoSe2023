%----------------------------------
%  TextCommands
%----------------------------------
%
%
%
%
%----------------------------------
%  common textCommands
%----------------------------------
% Information: OL bedeutet ohne Leerzeichen. Damit man dieses Command z. B. vor einem Komma oder vor einem anderen Zeichen verwenden kann. Dies ist ein Best-Practis von mir und hat sich sehr bewehrt.
% Allgemein hat es sich bewert alle Wörter die man häufig schreibt und wahrscheinlich falsch oder unterscheidlich schreibt, als Textcommand zu hinterlegen.
%
%
%
\renewcommand{\symheadingname}{Symbolverzeichnis}
\newcommand{\abbreHeadingName}{Abkürzungsverzeichnis}
\newcommand{\listEquationsName}{Formelverzeichnis}
\newcommand{\headingNameInternetSources}{Internetquellen}
\newcommand{\abbildungsverzeichnis}{Abbildungsverzeichnis}
\newcommand{\tabellenverzeichnis}{Tabellenverzeichnis}
\newcommand{\AppendixName}{Anhang}
\newcommand{\vglf}{Vgl.}
\newcommand{\pagef}{S. }
\newcommand{\os}{\mbox{o. S}}
\newcommand{\ojol}{\mbox{o. J.}}
\newcommand{\oj}{\ojol\ }
\newcommand{\og}{\mbox{o. g.}\ }
\newcommand{\uaol}{\mbox{u. a.}}
\newcommand{\ua}{\uaol\ }
\newcommand{\dah}{\mbox{d. h.}\ }
\newcommand{\zbol}{\mbox{z. B.}}
\newcommand{\zb}{\zbol\ }
\newcommand{\uamol}{unter anderem}
\newcommand{\uam}{\uamol\ }
\newcommand{\uanol}{unter anderen}%mit Leerzeichen
\newcommand{\uan}{\uanol\ }%mit Leerzeichen
\newcommand{\abbol}{Ab"-bil"-dung}
\newcommand{\abb}{\abbol\ }
\newcommand{\tabol}{Tabelle}
\newcommand{\tab}{\tabol\ }
\newcommand{\ggfol}{ggf.}
\newcommand{\ggf}{\ggfol\ }
\newcommand{\unodol}{und/oder}
\newcommand{\unod}{\unodol\ }
\newcommand{\bzwol}{bzw.}
\newcommand{\bzw}{\bzwol\ }
\newcommand{\sogol}{sog.}
\newcommand{\sog}{\sogol\ }
\newcommand{\sool}{\mbox{s. o.}}
\newcommand{\so}{\sool\ }
\newcommand{\solol}{siehe oben}
\newcommand{\sol}{\solol\ }

%----------------------------------
% project individual textCommands
%----------------------------------
\newcommand{\nonbreakdash}{"~}