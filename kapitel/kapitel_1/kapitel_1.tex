\newpage


\section[Grundlagen der Kryptografie - Kryptografische Verfahren und Algorithmen]{Grundlagen der \gls{kryptografie}\index{Kryptografie} - \glsdisp{kryptografie}{Kryptografische} Verfahren und \glspl{algorithmus}}\label{sec:grundlagen-der-kryptografie}


In der digitalen Welt spielt die \Gls{kryptografie}\index{Kryptografie} eine entscheidende Rolle.
Sie sichert Informationen durch Ver- und Entschlüsselung, um Vertraulichkeit, Integrität und Authentizität zu gewährleisten.
\glsdisp{kryptografie}{kryptografische}\index{Kryptografie} \glspl{algorithmus} dienen als grundlegende Bausteine, um Daten in eine unlesbare Form zu bringen und nur autorisierten Empfängern zugänglich zu machen.

Dieser Abschnitt gibt einen Überblick über die Grundlagen der \Gls{kryptografie}\index{Kryptografie} und verschiedene Arten von \glspl{algorithmus}, die in Webanwendungen und in der Informationssicherheit verwendet werden.
Es vermittelt die Prinzipien und Konzepte, um das Verständnis für die Bedeutung und den Einsatz von \Gls{kryptografie}\index{Kryptografie} in der vernetzten Welt zu stärken.


\subsection[Symmetrische Verschlüsselungsalgorithmen]{Symmetrische \glsdisp{algorithmus}{Verschlüsselungsalgorithmen} - \acf{DES}}\label{subsec:symmetrsiche-algorithmen}
Bei symmetrischen Verschlüsselungsalgorithmen\index{Kryptografie!Symmetrische Verschlüsselung} wird, wie in \autoref{fig:symmetricalEncoding}\autocite{Chapter211:online} dargestellt, derselbe Schlüssel zum Ver\nonbreakdash und Entschlüsseln einer Nachricht verwendet.
Dies ermöglicht eine einfache und schnelle Kommunikation, führt jedoch dazu, dass die Geheimhaltung des Schlüssels schnell zu einer Sicherheitslücke\index{Sicherheit} führen kann.

\begin{figure}[htbp]
    \includegraphics[width=1\linewidth]{abbildungen/symmetricEncoding}
    \centering
    \caption[
        Schaubild einer symmetrischen Verschlüsselung]{Schaubild einer symmetrischen Verschlüsselung\footnotemark}
    \label{fig:symmetricalEncoding}
\end{figure}\ \footnotetext{\cite{Chapter211:online}}


Symmetrische Verschlüsselungsverfahren\index{Kryptografie!Symmetrische Verschlüsselung} werden deshalb dann als sicher angesehen, wenn man ohne den Schlüssel den Klartext nicht aus dem verschlüsselten Text ermitteln kann.\autocite[\pagef~5]{kryptographische-algorithmen}

\ac{DES}, oft auch \ac{DEA}, wurde 1975 im \textit{Federal Register} der USA veröffentlicht und als Kooperation zwischen dem \ac{NIST}, der \ac{NSA} und \ac{IBM} entwickelt.\autocite[\pagef~232]{nsa-meyer}

\ac{DES} ist ein Blockchiffre-Verfahren, was bedeutet, dass die Daten in immer gleich großen Blöcken, hier in Blöcken von 64 Zeichen, und mit einer immer gleichen Schlüssellänge, hier 56 Bit, verschlüsselt werden.\autocite[\pagef~6]{kryptographische-algorithmen}


\subsection[Asymmetrische Algorithmen]{Asymmetrische \glspl{algorithmus}}\label{subsec:asymmetrische-algorithmen}
Asymmetrische Verschlüsselungsverfahren\index{Kryptografie!Asymmetrische Verschlüsselung} verwenden, im Gegensatz zu den Symmetrischen\index{Kryptografie!Symmetrische Verschlüsselung} (\solol), zwei Schlüssel.
Einen \gls{publicKey}, zum Verschlüsseln der Daten und einen \gls{privateKey}, zum Entschlüsseln.
Dabei muss ausschließlich der \gls{privateKey} geheim gehalten werden.


\subsubsection[RSA-Verfahren]{\acs{RSA}\nonbreakdash Verfahren}\label{subsubsec:rsa-verfahren}

\ac{RSA} ist das erste entwickelte \gls{publicKeyEncoding} und besitzt auch heute noch eine große Relevanz.\autocite[\pagef~168]{buchmann-einfuhrung-2016}

\paragraph[Schlüsselerzeugung]{Schlüsselerzeugung}\label{par:schluesselerzeugung}
Das Generieren des \ac{RSA}-Schlüsselpaars benötigt einige verschiedene Zahlen.
Der Sicherheitsparameter $k \in \mathbb{N}$ gibt die Größe des Produkts der beiden gewählten Primzahlen für die Verschlüsselung an.
Zwei statistisch unabhängige, zufällige Primzahlen $p$ und $q$ werden ausgewählt, um den \ac{RSA}-Modul n zu bilden.
Dieser Wert wird später für Ver- und Entschlüsselung verwendet und berechnet sich als $n = p*q$.

Zusätzlich wird eine natürliche, ungerade Zahl $e$ gewählt, für die

\begin{equation}
    1 < e < \varphi(n) = (p - 1)(q - 1)\ \text{und}\ \gcd(e, (p-1)(q-1)) = 1\label{eq:equation2}
\end{equation}
gilt und daraus mit den folgenden Bedingungen
\begin{equation}
    1 < d < (p-1)(q-1)\ \text{und}\ d*e \equiv 1\mod(p-1)(q-1)\label{eq:equation3}
\end{equation}
eine weitere Zahl \(d \in \mathbb{N}\) gebildet.

Da $\gcd(e, (p-1)(q-1)) = 1$ gilt, existiert eine solche Zahl $d$ definitiv.
Berechnet werden kann sie mit dem \glspl{extendedEuklidAlgorithm}.
Der \gls{publicKey} bildet sich dabei aus dem Paar $(e, n)$, der \gls{privateKey} aus der Zahl $d$.\autocite[\pagef~169]{buchmann-einfuhrung-2016}

Damit \ac{RSA} eine sichere Verschlüsselung ermöglichen kann, müssen die beiden Primzahlen $p$ und $q$ passend gewählt werden.
Dafür ist es üblich, dass $k$ als eine gerade, mindestens $2^{10}$\nonbreakdash Bit lange Zahl gewählt wird.\autocite[\pagef~169]{buchmann-einfuhrung-2016}

\paragraph{Verschlüsselung}\label{par:verschluesselung}
Um mit dem \ac{RSA} \gls{algorithmus} eine Nachricht zu verschlüsseln, wird der \gls{publicKey} $(e, n)$ benötigt.
Aus einem \gls{klartext} \(m \in \mathbb{Z}_m\) mit \(\mathbb{Z}_m\) als \ac{RSA}\nonbreakdash\gls{klartextraum} erhält man den verschlüsselten Text $c$ mit
\begin{equation}
    c = m^e\mod n\label{eq:equation4}
\end{equation}
$m$ kann wieder rekonstruiert werden mit
\begin{equation}
    m = c^d \mod n\label{eq:equation5}
\end{equation}
wobei $c$ der zuvor erhaltene verschlüsselte Text, $d$ der \gls{privateKey} und $n$ ein Teil des \glsdisp{publicKey}{public Keys} ist.\autocite[\pagef~6]{rsa-encryption}

\subsubsection[Hashfunktionen]{Hashfunktionen — \acf{SHA}}\label{subsubsec:hash-funktion}
Im Allgemeinen sind Hashfunktionen \glspl{algorithmus}, die einen Text beliebiger Länge zu einem neuen Text mit vorgegebener Länge komprimieren\autocite[\pagef~15]{anal-des-hash-function-2003}.
Sie werden mithilfe von \sog \glspl{compressfunc} generiert.

Damit \glsdisp{hashfunc}{Hash}- und \glspl{compressfunc} in der \gls{kryptografie}\index{Kryptografie} zur Authentifizierung, wie \zb zur Speicherung von Passwörtern genutzt werden können, müssen sie noch verschiedene Kriterien erfüllen.
Diese werden folgend erklärt:

\begin{definition}\label{def:hashfunktion}
    Eine Einweghashfunktion ist eine Funktion $h$, die folgende Bedingungen erfüllt\autocite[\vglf][\pagef~17]{anal-des-hash-function-2003}:
    \begin{enumerate}
        \item Die Beschreibung von $h$ muss öffentlich bekannt sein und sollte keine geheimen Informationen erfordern (Erweiterung des Kerckhoff\textquotesingle schen Prinzips\autocite[]{petitcolas-information-nodate}).
        \item Das Argument $X$ kann von beliebiger Länge sein und das Ergebnis $h(X)$ hat eine feste Länge von $n$-Bits (mit $n \geq64$).
        \item Für gegebene $h$ und $X$, muss die Berechnung von $h(X)$ einfach\footnotemark\ sein.
        \item Die Hashfunktion muss in dem Sinne monodirektional sein, dass es bei einem $Y$ im Abbild von $h$ schwer\footnotemark[\value{footnote}] ist, aus einer Nachricht einen bestimmten \gls{hashwert} zu generieren, und dass es schwer\footnotemark[\value{footnote}] ist, zwei Nachrichten zu finden, die den gleichen \gls{hashwert} teilen.
    \end{enumerate}
    \footnotetext{\enquote{einfach} und \enquote{schwer} sind hier im kryptografischen Sinne zu verstehen und beziehen sich auf das Zusammenspiel von Laufzeit und Rechenaufwand eines \gls{algorithmus}\textquotesingle}
\end{definition}

\autoref{def:hashfunktion} verwendet die Begriffe \enquote{einfach} und \enquote{schwer}, da heutzutage noch keine \glspl{algorithmus} bekannt sind, die eine \glsdisp{hashfunc}{Einweghashfunktion} schnell genug umkehren können.\autocite[\pagef~234]{buchmann-einfuhrung-2016}

\acfp{SHA} sind verschiedene kryptologische Hashfunktionen und eine Modifikation des \gls{MD5}, welche zur Berechnung eines Prüfwertes für beliebige Nachrichten dienen und unter anderem die Grundlage zur Erstellung einer digitalen Signatur, genauer erläutert in \autoref{subsubsec:digitale-signaturen-und-zertifikate}, sind\autocite[]{WhatisSH81:online}.

2012 wurde \ac{SHA}-3\footnote{Auch als \textit{Keccak} bezeichnet}  vom \ac{NIST} standardisiert und wird heute als sicher angesehen\autocite[\pagef~239]{buchmann-einfuhrung-2016}, aber auch die \glspl{algorithmus} unter \ac{SHA}-2 sind heute stark verbreitet.
\ac{SHA}-2 und \ac{SHA}-3 bezeichnen nicht einzelne \glspl{algorithmus} sondern \glsdisp{algorithmus}{Algorithmusgruppen}, deren zugrunde liegende \glspl{algorithmus} sich primär in der Länge des ausgegebenen Prüfwertes unterscheiden.
Dabei werden die \glspl{algorithmus} \ac{SHA}-256 und \ac{SHA}-512 am häufigsten genutzt.