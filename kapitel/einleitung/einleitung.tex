\section{Einleitung}\label{sec:einleitung}
Die Sicherheit von Daten in der digitalen Welt ist von größter Bedeutung. 
Das Internet und Webanwendungen haben unser Leben verbessert, aber auch neue Herausforderungen in Bezug auf Datensicherheit gebracht.

\Gls{kryptografie}index{Kryptografie} ermöglicht durch Verschlüsselungs- und Signaturverfahren eine sichere Übertragung und Speicherung von Daten.
Sie sichert Daten und Informationen, die über das Internet oder andere Netzwerke übertragen werden, mit mathematischen Verfahren. 
Verschlüsselung gewährleistet, dass nur autorisierte Personen die Informationen lesen können, während digitale Signaturen Manipulation erkennbar machen.


\subsection{Zielsetzung}\label{subsec:zielsetzung}
In dieser Arbeit wird untersucht, wie Verschlüsselung und digitale Signaturen als mathematische Methoden eingesetzt werden, um Webanwendungen zu sichern.
Webanwendungen werden immer häufiger zur Speicherung und Verarbeitung von verschiedensten Daten und Informationen verwendet und stellen daher ein attraktives Ziel für Angreifer dar.
Durch den Einsatz von \gls{kryptografie}index{Kryptografie} kann die Sicherheit von Webanwendungen erhöht werden.

Dabei wird versucht, die Forschungsfrage \enquote{\myForschungsfrage} anhand einer Literaturanalyse zu beantworten.


\subsection{Aufbau der Arbeit}\label{subsec:aufbau-der-arbeit}
 \autoref{sec:grundlagen-der-kryptografie} erklärt einige grundlegende Konzepte der \gls{kryptografie}index{Kryptografie}, und stellt verschiedene \glsdisp{kryptografie}{kryptografische}\index{kryptografie} Verfahren und \glspl{algorithmus} dar.
\autoref{sec:anwendung_von_kryptografie_in_webanwendungen} befasst sich mit verschiedenen Möglichkeiten, Verschlüsselungsverfahren in Webanwendungen einzubinden und die Sicherheit von Daten zu gewährleisten, während  \autoref{sec:Herausforderung-bei-der-implementierung-von-kryptografie-in-webanwendungen} Probleme und Schwierigkeiten, die während und nach der Implementierung dieser Verfahren auftreten können, erläutert.