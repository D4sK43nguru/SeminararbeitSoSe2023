\section{Einleitung}
In der heutigen digitalen Welt ist die Sicherheit von Daten und Informationen von größter Bedeutung. Das Internet und Webanwendungen haben unser Leben erleichtert und verbessert, aber auch neue Herausforderungen in Bezug auf Datensicherheit und Datenschutz mit sich gebracht. Durch den Einsatz von Verschlüsselungs- und Signaturverfahren können Daten und Informationen sicherer übertragen und gespeichert werden. Das ist das Ziel der Kryptographie.

Kryptographie ist die Wissenschaft der Ver- und Entschlüsselung von Informationen. Sie umfasst mathematische Verfahren zur Sicherung von Daten und Informationen, die über das Internet oder andere Netzwerke übertragen werden. Die Verschlüsselung stellt sicher, dass die Informationen nur von autorisierten Personen gelesen werden können, während digitale Signaturen gewährleisten, dass die Informationen nicht manipuliert wurden.

\subsection{Zielsetzung}
In dieser Arbeit wird untersucht, wie mathematische Methoden, wie die Verschlüsselung und die digitale Signatur, eingesetzt werden, um Web-Anwendungen zu sichern. Webanwendungen werden immer häufiger zur Speicherung und Verarbeitung von Daten und Informationen verwendet und stellen daher ein attraktives Ziel für Angreifer dar. Durch den Einsatz von Kryptographie kann die Sicherheit von Webanwendungen erhöht werden.

Dabei wird versucht, die Forschungsfrage \enquote{Wie sehr tragen die aktuell genutzten \glsdisp{kryptographie}{kryptographischen} Methoden zur Sicherheit von Daten in Webanwendungen bei?} anhand einer Literaturanalyse zu beantworten.


\subsection{Aufbau der Arbeit}
Dafür werden zunächst ein paar Grundbegriffe und grundlegende Konzepte der \gls{kryptographie} erklärt, sowie verschiedene \glsdisp{kryptographie}{kryptographische} Verfahren und \glsdisp{algorithmus} dargestellt. 
\autoref{sec:anwendung_von_kryptographie_in_webanwendungen} befasst sich mit verschiedenen Möglichkeiten, Verschlüsselungsverfahren in Webanwendungen einzubinden und die Sicherheit von Daten zu gewährleisten, während  \autoref{sec:Herausforderung-bei-der-implementierung-von-kryptographie-in-webanwendungen} Probleme und Schwierigkeiten, die während und nach der Implementierung dieser Verfahren auftreten können erläutert.