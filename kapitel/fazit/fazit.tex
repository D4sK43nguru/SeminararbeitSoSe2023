\newpage
\section{Fazit}\label{sec:fazit}

In der vorliegenden Arbeit wurde das Thema \Gls{kryptografie}\index{Kryptografie} in Webanwendungen untersucht.
Folgende Fragestellung lag der Untersuchung zugrunde: \myForschungsfrage\

Mithilfe einer ausgiebigen Literaturanalyse konnten dafür verschiedene \glsdisp{kryptografie}{kryptografische} \glspl{algorithmus} untersucht und daraus gebildete Techniken analysiert werden.

Die Sicherheit von Daten in Webanwendungen wird drastisch durch ein Zusammenspiel verschiedener einfacher \glspl{algorithmus} erhöht und noch weiter durch daraus gebildete Techniken, wie \zb digitale Zertifikate, verstärkt.
Diese Arbeit zeigt auch, dass ein Großteil der Nutzer von Webanwendungen nur ein Passwort nutzt, um ihre Daten zu sichern, sowie dass viele Nutzer ihre Passwörter nur unregelmäßig wechseln.
Die Kombination dieser beiden Fakten stellt bei den meisten Systemen ein großes Sicherheitsrisiko dar, da viele Systeme \zb nicht ausreichend auf die Möglichkeit hinweisen, \ac{2FA}\nonbreakdash Services zu nutzen und einige Systeme Daten unverschlüsselt übertragen oder speichern.
Dies ist jedoch ein rückgängiger Trend.

Die Analyse der verschiedenen \glsdisp{kryptografie}{kryptografischen} Methoden im Zusammenhang mit den Problemen und Hindernissen, die diese auslösen, zeigt, dass es schwierig ist, eine Webanwendung ausgiebig zu sichern.
Jedoch gibt es mehrere verschiedene Methoden, die Daten zu sichern, die auch miteinander agieren können und sich gegenseitig verstärken können.
So ist es möglich, \zb Passwörter zu \glsdisp{hashfunc}{hashen}, bevor sie in einer Datenbank gespeichert werden und zudem noch die Verbindung mit einem \ac{TLS}\nonbreakdash Zertifikat abzusichern.
Das Ziel dieser Arbeit, den Nutzen verschiedener \glsdisp{kryptografie}{kryptografischer} Methoden vor- und ihre Auswirkungen darzustellen, wurde somit erreicht.

Die Untersuchung der einzelnen Möglichkeiten zeigt, dass beim Absichern von Daten oder Verbindungen auf viele einzelne Aspekte zu achten ist.
Diese einzeln genauer zu untersuchen und Möglichkeiten zu erarbeiten, die Sicherheit von Webanwendungen zu vereinfachen könnte für eine weiterführende Auseinandersetzung interessant sein.

Zudem beschäftigt sich die vorliegende Arbeit ausschließlich mit den Auswirkungen der einzelnen Funktionen und nicht mit dem prinzipiellen Arbeitsablauf.
Hier bleibt im Detail zu klären, wie die einzelnen Funktionen arbeiten und wie man sie entweder einzeln oder im Zusammenspiel mit anderen Funktionen optimieren kann.